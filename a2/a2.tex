\documentclass{article}

\usepackage{mathtools}
\usepackage[margin=1in]{geometry}
\usepackage{float}
\usepackage{graphicx}
\usepackage{xfrac}
\usepackage{enumitem}
\usepackage{multirow}

\usepackage{proof}
\usepackage{amssymb}
\usepackage{latexsym}

\author{Guillaume Labranche (260585371)}
\title{COMP 527 -- Logic and Computation\\Homework 2}
\date{due on 11 February 2016}

\newcommand{\D}{\mathcal{D}}
\newcommand{\E}{\mathcal{E}}
\newcommand{\F}{\mathcal{F}}
\newcommand{\then}{\supset}
\newcommand{\proves}{\vdash}
\newcommand{\GV}{\Gamma \proves}
%\newcommand{\fst}{\text{fst }}
%\newcommand{\snd}{\text{snd }}
\DeclareMathOperator{\fst}{fst}
\DeclareMathOperator{\snd}{snd}
\DeclareMathOperator{\inl}{inl}
\DeclareMathOperator{\inr}{inr}
\DeclareMathOperator{\abort}{abort}
\DeclareMathOperator{\tasdasfd}{true}
%\renewcommand{\t}{\tasdasfd{}}
\renewcommand{\t}{\text{}}
\begin{document}

\maketitle

\begin{enumerate}[label=\textbf{Exercise \arabic*}]


\item If and only if

\begin{enumerate}[label=\textbf{Task \arabic*},leftmargin=1em]
\item Annotated introduction and elimination rules for $\equiv$.
\[
\infer[\equiv I^{u,v}]
  {\langle \lambda x \in A.M,\lambda y \in B.N \rangle A \equiv B \t}
  {\infer*{M:B\t}{\infer[u]{x:A\t}{}} & \infer*{N:A\t}{\infer[v]{y:B\t}{}}}
\]
\begin{tabular}{cc}
$$
\infer[\equiv E_L]
  {(\fst M) x : B \t}
  {M:A \equiv B \t & x : A \t}
$$
&
$$
\infer[\equiv E_R]
  {(\snd M) x : A \t}
  {M:A \equiv B \t & x : B \t}
$$
\end{tabular}

\item Local reduction: %soundness
\[
\infer[\equiv E_L]
  { (\fst \langle \lambda x \in A.M,\lambda y \in B.N \rangle) w : B \t}
  {\infer[\equiv I^{u,v}]
    {\langle \lambda x \in A.M,\lambda y \in B.N \rangle A \equiv B \t}
    {\deduce[\mathcal{D}]{M:B\t}{\infer[u]{x:A\t}{}}
     & \deduce[\mathcal{E}]{N:A\t}{\infer[v]{y:B\t}{}}
     }
  & \deduce[\mathcal{F}]{w:A \t}{}
  }
\Rightarrow_R
\deduce[{[}\mathcal{F}/x{]}\mathcal{D}]
  {M:B \t}
  {}
\]

\[
\infer[\equiv E_L]
  { (\snd \langle \lambda x \in A.M,\lambda y \in B.N \rangle) w : A \t}
  {\infer[\equiv I^{u,v}]
    {\langle \lambda x \in A.M,\lambda y \in B.N \rangle A \equiv B \t}
    {\deduce[\mathcal{D}]{M:B\t}{\infer[u]{x:A\t}{}}
     & \deduce[\mathcal{E}]{N:A\t}{\infer[v]{y:B\t}{}}
     }
  & \deduce[\mathcal{F}]{w:B \t}{}
  }
\Rightarrow_R
\deduce[{[}\mathcal{F}/y{]}\mathcal{E}]
  {N:A \t}
  {}
\]

Local expansion: %completeness
\[
\deduce[\mathcal{D}]{M : A \equiv B \t}{}
\Rightarrow_E
\infer[\equiv I^{u,v}]
  {\langle \fst M, \snd M \rangle : A \equiv B \t}
  {\infer[\equiv E_L]
    {(\fst M)x : B \t}
    {\deduce[\mathcal{D}]{M : A \equiv B \t}{}
    & \infer[u]{x:A \t}{}}
  &\infer[\equiv E_L]
    {(\snd M)y : A \t}
    {\deduce[\mathcal{D}]{M : A \equiv B \t}{}
    & \infer[u]{y:B \t}{}}
  }
\]

\item Here are three cases:

$$\D = \infer[\equiv I^{u, v}]
  {\Gamma, x:A, \Gamma' \vdash B_1 \equiv B_2}
  {\Gamma, x:A, \Gamma', u:B_2 \vdash B_1 & \Gamma, x:A, \Gamma', v:B_1 \vdash B_2}
$$

$$\D = \infer[\equiv E_l]
  {\Gamma, x:A, \Gamma' \vdash B_2}
  {\Gamma, x:A, \Gamma' \vdash B_1 \equiv B_2 & \Gamma, x:A, \Gamma' \vdash B_1}
$$

$$\D = \infer[\equiv E_r]
  {\Gamma, x:A, \Gamma' \vdash B_1}
  {\Gamma, x:A, \Gamma' \vdash B_1 \equiv B_2 & \Gamma, x:A, \Gamma' \vdash B_2}
$$

\item 
\[
\infer[\equiv I^{u,v}]
  {\Gamma^\downarrow \proves \langle \lambda u \in A.M,\lambda v \in B.N \rangle A \equiv B \uparrow}
  {\Gamma^\downarrow, u:A {\downarrow} \proves M:B \uparrow
  &\Gamma^\downarrow, v:B {\downarrow} \proves N:A \uparrow}
\]
\begin{tabular}{cc}
$$
\infer[\equiv E_L]
  { \Gamma^\downarrow \proves (\fst M) x : B \downarrow }
  { \Gamma^\downarrow \proves M:A \equiv B \downarrow
  & \Gamma^\downarrow \proves x : A \uparrow }
$$
&
$$
\infer[\equiv E_R]
  { \Gamma^\downarrow \proves (\snd M) x : A \downarrow }
  { \Gamma^\downarrow \proves M:A \equiv B \downarrow
  & \Gamma^\downarrow \proves x : B \uparrow }
$$
\end{tabular}

\end{enumerate}

\item We prove this by structural induction on the proof tree $\mathcal{D}$.

\begin{itemize}
\item First we show the base cases:
\begin{tabular}{cc}
$\D = \infer[\top I]{\GV () : \top}{}$
&
$\D' = \infer[\top I]{\GV () : \top}{}$
\end{tabular}
\begin{align*}
\top &= \top && \text{by definition}
\end{align*}


\item case
\begin{tabular}{cc}
$\D = \infer[\land I]
  {\GV \langle M,N \rangle : A_1 \land B_1}
  {\deduce{\GV M:A_1}{\E_1}
  &\deduce{\GV N:B_1}{\E_2} }$
&
$\D' = \infer[\land I]
  {\GV \langle M,N \rangle : A_2 \land B_2}
  {\deduce{\GV M:A_2}{\F_1}
  &\deduce{\GV N:B_2}{\F_2} }$
\end{tabular}
\begin{align*}
A_1 &= A_2 && \text{by IH on $\E_1$ and $\F_1$} \\
B_1 &= B_2 && \text{by IH on $\E_2$ and $\F_2$} \\
\GV \langle M,N \rangle &: A_1 \land B_1 && \text{by $\land I$ on $\F_1$ and $\F_2$} \\
A_1 \land B_1 &= A_2 \land B_2 && \text{because we were able to derive $\GV \langle M,N \rangle : A_1 \land B_1$ from $\F_1,\F_2$}
\end{align*}


\item case
\begin{tabular}{cc}
$\D = \infer[\land E_l]
  {\GV \fst M:A_1}
  {\deduce{\GV M : A_1 \land B}{\E}}$
&
$\D' = \infer[\land E_l]
  {\GV \fst M:A_2}
  {\deduce{\GV M : A_2 \land B}{\E'}}$
\end{tabular}
\begin{align*}
A_1 \land B &= A_2 \land B && \text{by IH on $\E$ and $\E'$} \\
\GV \fst M &: A_1 && \text{by $\land E_l$ on $\E'$} \\
A_1&=A_2 && \text{because we derived $\GV \fst M:A_1$ from $\E'$}
\end{align*}


\item case
\begin{tabular}{cc}
$\D = \infer[\land E_r]
  {\GV \snd M:B_1}
  {\deduce{\GV M : A \land B_1}{\E}}$
&
$\D' = \infer[\land E_r]
  {\GV \snd M:B_2}
  {\deduce{\GV M : A \land B_2}{\E'}}$
\end{tabular}
\begin{align*}
A \land B_1 &= A \land B_2 && \text{by IH on $\E$ and $\E'$} \\
\GV \snd M &: B_1 && \text{by $\land E_r$ on $\E'$} \\
B_1&=B_2 && \text{because we derived $\GV \snd M:B_1$ from $\E'$}
\end{align*}


\item case
\begin{tabular}{cc}
$\D = \infer[\lor I_l]
  {\GV \inl^B M: A_1 \lor B}
  {\deduce{\GV M : A_1}{\E}}$
&
$\D' = \infer[\lor I_l]
  {\GV \inl^B M: A_2 \lor B}
  {\deduce{\GV M : A_2}{\E'}}$
\end{tabular}
\begin{align*}
A_1 &= A_2 && \text{by IH on $\E$ and $\E'$} \\
\GV \inl^B M&: A_1 \lor B && \text{by $\lor I_l$ on $\E'$} \\
A_1 \lor B &= A_2 \lor B && \text{because we derived $\GV \inl^B M: A_1 \lor B$ from $\E'$}
\end{align*}


\item case
\begin{tabular}{cc}
$\D = \infer[\lor I_r]
  {\GV \inr^A M: A \lor B_1}
  {\deduce{\GV M : B_1}{\E}}$
&
$\D' = \infer[\lor I_r]
  {\GV \inr^A M: A \lor B_2}
  {\deduce{\GV M : B_2}{\E'}}$
\end{tabular}
\begin{align*}
B_1 &= B_2 && \text{by IH on $\E$ and $\E'$} \\
\GV \inl^A M&: A \lor B_1 && \text{by $\lor I_r$ on $\E'$} \\
A \lor B_1 &= A \lor B_2 && \text{because we derived $\GV \inr^A M: A \lor B_1$ from $\E'$}
\end{align*}


\item case
\begin{tabular}{cc}
$\D = \infer[\then I^u]
  {\GV \lambda u \in A_1 . M : A_1 \then B_1}
  {\deduce{\Gamma,u:A_1 \proves M:B_1}{\E}}$
&
$\D' = \infer[\then I^u]
  {\GV \lambda u \in A_2 . M : A_2 \then B_2}
  {\deduce{\Gamma,u:A_2 \proves M:B_2}{\E'}}$
\end{tabular}
\begin{align*}
A_1 &= A_2 && \text{by IH on $\E$ and $\E'$} \\
B_1 &= B_2 && \text{by IH on $\E$ and $\E'$} \\
\Gamma,u:A_1 &\proves M:B_1 && \text{by $\then I$ on $\E'$} \\
A_1 \then B_1 &= A_2 \then B_2 && \text{because we derived $A_1 \then B_1$ from $\E'$}
\end{align*}

The other cases follow a very similar argument.
\end{itemize}

\item

\begin{enumerate}[label=\arabic*.,leftmargin=1em]
\item Proof and term given as \texttt{conj1} in \texttt{ex3.tut}.
\item The proof breaks down because doing an $\exists E^{x,p}$ gives us $x:\tau$ and $A\then P(x)$ yet in order to prove $\forall y\in\tau.P(y)$ we need an arbitrary $y\in\tau$.
\item Same reason as the previous conjecture.
\item Proof and term given as \texttt{conj4} in \texttt{ex3.tut}.
\item As shown in the course notes, we need a witness for $x\in\tau$ such that $\neg B(x)$ yet we know nothing about the domain $\tau$. Therefore we cannot produce one just fdrom the assumption $\neg(\forall x\in\tau.B(x))$.
\item Proof and term given as \texttt{conj6} in \texttt{ex3.tut}.
\end{enumerate}

\item Provable
\begin{enumerate}[label=\arabic*.,leftmargin=1em]
\item Proof and term given as \texttt{conj1} in \texttt{ex4.tut}.
\item Proof and term given as \texttt{conj2} in \texttt{ex4.tut}.
\item Proof and term given as \texttt{conj3} in \texttt{ex4.tut}.
\end{enumerate}

\end{enumerate}

\end{document}