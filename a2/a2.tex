\documentclass{article}

\usepackage{mathtools}
\usepackage[margin=1in]{geometry}
\usepackage{float}
\usepackage{graphicx}
\usepackage{xfrac}
\usepackage{enumitem}
\usepackage{multirow}

\usepackage{proof}
\usepackage{amssymb}
\usepackage{latexsym}

\author{Guillaume Labranche (260585371)}
\title{COMP 527 -- Logic and Computation\\Homework 2}
\date{due on 11 February 2016}

\newcommand{\R}{\mathbb{R}}
\newcommand{\F}{\mathbb{F}}
%\renewcommand{\t}{\text{ true}}
\renewcommand{\t}{\text{}}
\newcommand{\fst}{\text{fst }}
\newcommand{\snd}{\text{snd }}
\begin{document}

\maketitle

\begin{enumerate}[label=\textbf{Exercise \arabic*}]


\item If and only if

\begin{enumerate}[label=\textbf{Task \arabic*},leftmargin=1em]
\item Annotated introduction and elimination rules for $\equiv$.
\[
\infer[\equiv I^{u,v}]
  {\langle \lambda x \in A.M,\lambda y \in B.N \rangle A \equiv B \t}
  {\infer*{M:B\t}{\infer[u]{x:A\t}{}} & \infer*{N:A\t}{\infer[v]{y:B\t}{}}}
\]
\begin{tabular}{cc}
$$
\infer[\equiv E_L]
  {(\fst M) x : B \t}
  {M:A \equiv B \t & x : A \t}
$$
&
$$
\infer[\equiv E_L]
  {(\snd M) x : A \t}
  {M:A \equiv B \t & x : B \t}
$$
\end{tabular}

\item Local reduction %soundness
\[
\infer[\equiv E_L]
  { (\fst \langle \lambda x \in A.M,\lambda y \in B.N \rangle) w : B \t}
  {\infer[\equiv I^{u,v}]
    {\langle \lambda x \in A.M,\lambda y \in B.N \rangle A \equiv B \t}
    {\deduce[\mathcal{D}]{M:B\t}{\infer[u]{x:A\t}{}}
     & \deduce[\mathcal{E}]{N:A\t}{\infer[v]{y:B\t}{}}
     }
  & \deduce[\mathcal{F}]{w:A \t}{}
  }
\Rightarrow_R
\deduce[{[}\mathcal{F}/x{]}\mathcal{D}]
  {M:B \t}
  {}
\]

\[
\infer[\equiv E_L]
  { (\snd \langle \lambda x \in A.M,\lambda y \in B.N \rangle) w : A \t}
  {\infer[\equiv I^{u,v}]
    {\langle \lambda x \in A.M,\lambda y \in B.N \rangle A \equiv B \t}
    {\deduce[\mathcal{D}]{M:B\t}{\infer[u]{x:A\t}{}}
     & \deduce[\mathcal{E}]{N:A\t}{\infer[v]{y:B\t}{}}
     }
  & \deduce[\mathcal{F}]{w:B \t}{}
  }
\Rightarrow_R
\deduce[{[}\mathcal{F}/y{]}\mathcal{E}]
  {N:A \t}
  {}
\]

Local expansion %completeness
\[
\deduce[\mathcal{D}]{M : A \equiv B \t}{}
\Rightarrow_E
\infer[\equiv I^{u,v}]
  {\langle \fst M, \snd M \rangle : A \equiv B \t}
  {\infer[\equiv E_L]
    {(\fst M)x : B \t}
    {\deduce[\mathcal{D}]{M : A \equiv B \t}{}
    & \infer[u]{x:A \t}{}}
  &\infer[\equiv E_L]
    {(\snd M)y : A \t}
    {\deduce[\mathcal{D}]{M : A \equiv B \t}{}
    & \infer[u]{y:B \t}{}}
  }
\]

\item non
\item 6 pts
\end{enumerate}

\item We prove this by structural induction on the rules

\begin{tabular}{cc}
$$
\infer[\land I]
  {\Gamma \vdash \langle M,N \rangle : A\land B}
  {\Gamma \vdash M:A & \Gamma \vdash N:B}
$$
&$$
$$
\end{tabular}

\item

\begin{enumerate}[label=\arabic*.,leftmargin=1em]
\item Proof and term given as \texttt{conj1} in \texttt{ex3.tut}
\item The proof breaks down because doing an $\exists E^{x,p}$ gives us $x:\tau$ and $A\supset P(x)$ yet in order to prove $\forall y\in\tau.P(y)$ we need an arbitrary $y\in\tau$.
\item Not provable
\item Proof and term given as \texttt{conj4} in \texttt{ex3.tut}
\item As shown in the course notes, we need a witness for $x\in\tau$ such that $\neg B(x)$ yet we know nothing about the domain $\tau$. Therefore we cannot produce one just from the assumption $\neg(\forall x\in\tau.B(x))$.
\item Proof and term given as \texttt{conj6} in \texttt{ex3.tut}
\end{enumerate}

\item Provable
\begin{enumerate}[label=\arabic*.,leftmargin=1em]
\item Proof and term given as \texttt{conj1} in \texttt{ex4.tut}
\item Proof and term given as \texttt{conj2} in \texttt{ex4.tut}
\item Proof and term given as \texttt{conj3} in \texttt{ex4.tut}
\end{enumerate}

\end{enumerate}

\end{document}